 \documentclass[11pt]{article}
%\documentstyle[11pt]{article}
\usepackage{layout}
\usepackage{amssymb}
\usepackage{epsfig}
\usepackage{graphicx}
\usepackage{color}
%\usepackage{showkeys}
\usepackage{latexsym}

\setlength{\textwidth}{16.0cm} 
\setlength{\textheight}{21.0cm} 
\setlength{\evensidemargin}{ 0.5cm}
\setlength{\oddsidemargin} { 0.5cm} 
\setlength{\topmargin}{-0.5cm} \setlength{\baselineskip}  { 0.7cm}


\usepackage{amssymb}
\usepackage{algorithmicx}
\usepackage[ruled]{algorithm}
\usepackage{algpseudocode}
\usepackage{algpascal}
\usepackage{algc}

\alglanguage{pseudocode}
%%%%%%%%%%%%%%%%%%%%%%%%%%%%%%%%%%%%%%%%%%%%%%%%%%%%%%%%%%%%%%%%%%%%%%%%%%%%%%%%
%%%%%%%%%%%%%%%%%%%%

%\usepackage[counterclockwise]{rotating}
\usepackage{collectbox}

\makeatletter
\newcommand{\sqbox}{%
    \collectbox{%
        \@tempdima=\dimexpr\width-\totalheight\relax
        \ifdim\@tempdima<\z@
            \fbox{\hbox{\hspace{-.5\@tempdima}\BOXCONTENT\hspace{-.5\@tempdima}}}%
        \else
            \ht\collectedbox=\dimexpr\ht\collectedbox+.5\@tempdima\relax
            \dp\collectedbox=\dimexpr\dp\collectedbox+.5\@tempdima\relax
            \fbox{\BOXCONTENT}%
        \fi
    }%
}
\makeatother



\newenvironment{proof}[1][Proof]{\textbf{#1.} }{\ \rule{0.5em}{0.5em}}
\newcommand{\R}{I\!\!R}
\def\C{{\rm I\!C}}
\def\N{{\rm I\!N}}
\newtheorem{theorem}{Theorem}[section]
\newtheorem{acknowledgement}[theorem]{Acknowledgement}
%\newtheorem{algorithm}[theorem]{Algorithm}
\newtheorem{axiom}[theorem]{Axiom}
\newtheorem{case}[theorem]{Case}
\newtheorem{claim}[theorem]{Claim}
\newtheorem{conclusion}[theorem]{Conclusion}
\newtheorem{condition}[theorem]{Condition}
\newtheorem{conjecture}[theorem]{Conjecture}
\newtheorem{corollary}[theorem]{Corollary}
\newtheorem{criterion}[theorem]{Criterion}
\newtheorem{definition}[theorem]{Definition}
\newtheorem{example}[theorem]{Example}
\newtheorem{exercise}[theorem]{Exercise}
\newtheorem{lemma}[theorem]{Lemma}
\newtheorem{notation}[theorem]{Notation}
\newtheorem{problem}[theorem]{Problem}
\newtheorem{proposition}[theorem]{Proposition}
\newtheorem{remark}[theorem]{Remark}
\newtheorem{solution}[theorem]{Solution}
\newtheorem{summary}[theorem]{Summary}
\newcommand{\Frac}[2] {\frac{\textstyle #1} {\textstyle #2}}
\newcommand{\taumin}{\tau_{min}}
\newcommand{\taumax}{\tau_{max}}
\newcommand{\sigmin}{\sigma_{min}}
\newcommand{\sigmax}{\sigma_{max}}
 \newcommand{\halmos}{\hfill $\;\;\;\Box$\\}

\begin{document}

\title{Stabilized Schemes for Phase Fields Models}
\author{M. Brachet\thanks{Institut Elie Cartan de Lorraine, Universit\'e de Lorraine, Site de Metz, B\^at. A Ile du Saucy, F-57045 Metz Cedex 1, {\tt matthieu.brachet@math.univ-metz.fr}} \and J.-P. Chehab\thanks{
Laboratoire Amienois de Math\'ematiques Fondamentales et Appliqu\'ees (LAMFA), {\small UMR} 7352,
 Universit\'e de Picardie Jules Verne, 33 rue Saint Leu, 80039 Amiens France
 , ({\tt
 Jean-Paul.Chehab@u-picardie.fr}).} }


%\date{}


\maketitle

%
%



\begin{abstract}
\end{abstract}
\section{Introduction}
%%%%%%%%%%%%%%%%%%%%%%%%%%%%%%%%%%%%%%%%%%%%%%%%%
%
%RSS SCHEMES, PRINCIPLE PROPERTIES AND NEW EXTENTIONS
%
%%%%%%%%%%%%%%%%%%%%%%%%%%%%%%%%%%%%%%%%%%%%%%%%%
\section{The RSS-schemes for parabolic problems}
The forward Euler's scheme is known to be stable only for small time steps; this rectriction can be hard when considering heat-equation, the basic linear part of reaction-diffusion equations on which we focus here. This is due to the necessity of not allowing the expansion of high mode components which leads to the divergence of the scheme.
A way to overcome the lake of stability consist in a approximation of the Backward Euler's scheme as follows. Consider the time and space discretization of the heat equation
\begin{eqnarray}\label{BE_HEAT}
\Frac{u^{(k+1)}-u^{(k)}}{\Delta t}+Au^{(k+1)}=0,
\end{eqnarray}
wher $A$ is the stifness matrix, $\Delta t>0$, the time step; here $u^{(k)}$ is the approximation of the solution at time $t=k\Delta t$ in the spatial approximation space. To simplify the linear system that must be solved at each step,  we replace $Au^{(k+1)}$ by
$\tau B (u^{(k+1)}-u^{(k)}) +Au^{(k)}$, where $\tau\ge 0$ and where $B$ is a preconditioner of $A$. This leads to the so-called RSS scheme
\begin{eqnarray}\label{RSS_HEAT}
\Frac{u^{(k+1)}-u^{(k)}}{\Delta t}+\tau B (u^{(k+1)}-u^{(k)}) +Au^{(k)}=0.
\end{eqnarray}
Chosing a "good" and appropriate preconditioner for enhancing the stability of the scheme
(\ref{RSS_HEAT}) as respect to the forward Euler scheme is not {\it a priori} an easy task. Rewriting (\ref{RSS_HEAT}) as
\begin{eqnarray}\label{RSS_HEAT2}
\Frac{u^{(k+1)}-u^{(k)}}{\Delta t}+\tau B u^{(k+1)} +(A-\tau B)u^{(k)}=0.
\end{eqnarray}

\subsection{RSS-Schemes and stabilization}
Let $A$ and $B$ be two $n\times n$ symmetric positive definite matrices. We assume that there exist two strictly positive constant $\alpha$ and $beta$ such that
\begin{eqnarray}
\label{Hyp_H}
\alpha <Bu,u> \le <A u, u> \le \beta <Bu,u>, \ \forall u \in \R^n
\end{eqnarray}
We first recall the following basic  result \cite{BrachetChehabJSC}
\begin{proposition}
Assume that $A$ and $B$ are two SPD matrices.
Under hypothesis (\ref{Hyp_H}), we have the following stability conditions:
\begin{itemize}
\item If $\tau\ge \Frac{\beta}{2}$, the schemes (\ref{RSS_ADI1}) and (\ref{RSS_ADI2}) are unconditionally stable (i.e. stable $\forall \ \Delta t >0$)
\item If $\tau < \Frac{\beta}{2}$, then the scheme is stable for
$
0<\Delta t < \Frac{2}{\left(1-\Frac{2\tau}{\beta}\right)\rho(A)}.
$
\end{itemize} 
\label{RSS_Stab_lin}
\end{proposition}
Of course we can consider second order schemes such as Gear's and apply the RSS stabilization. We can prove similar stability results:
\begin{proposition}
Consider the RSS-scheme derived from Gear's method 
\begin{eqnarray*}\label{RSS_GEAR}
\Frac{1}{2\Delta t}(3u^{(k+1)}-4u^{(k)}+u^{(k-1)})
+\tau B(u^{(k+1)}-u^{(k)})+Au^k=0
\end{eqnarray*}
We have the following stability conditions
\begin{itemize}
\item If $\tau\ge \Frac{\beta}{2}$, then (\ref{RSS_GEAR}) is unconditionally stable
\item If $\tau < \Frac{\beta}{2}$, then (\ref{RSS_GEAR}) is table when
$$
0<\Delta t < \Frac{2}{\rho(A)(1-\Frac{2\tau}{\beta})}
$$
\end{itemize}
\end{proposition}
\begin{proof}
We start from the identity
$$
\begin{array}{ll}
<3u^{(k+1)}-4u^{(k)}+u^{(k-1)},u^{(k+1)}-u^{(k)}>
&=2\|u^{(k+1)}-u^{(k)}\|^2 +\Frac{1}{2}(\|u^{(k+1)}-u^{(k)}\|^2\\
&-\|u^{(k)}-u^{(k-1)}\|^2
+\|u^{(k+1)}-2u^{(k)}+u^{(k-1)}\|^2)
\end{array}
$$
We now take the scalar product of each term of (\ref{RSS_GEAR}) with $u^{(k+1)}-u^{(k)}$ and obtain
$$
\begin{array}{l}
2\|u^{(k+1)}-u^{(k)}\|^2 +\Frac{1}{2}(\|u^{(k+1)}-u^{(k)}\|^2-\|u^{(k)}-u^{(k-1)}\|^2
+\|u^{(k+1)}-2u^{(k)}+u^{(k-1)}\|^2)\\
+2 \Delta t \left(\tau <B(u^{(k+1)}-u^{(k)}),u^{(k+1)}-u^{(k)}> +<Au^{(k)},u^{(k+1)}-u^{(k)}>\right)=0
\end{array}
$$
Using the parallelogram identity on the last term, we find
$$
\begin{array}{l}
2\|u^{(k+1)}-u^{(k)}\|^2 +\Frac{1}{2}(\|u^{(k+1)}-u^{(k)}\|^2-\|u^{(k)}-u^{(k-1)}\|^2
+\|u^{(k+1)}-2u^{(k)}+u^{(k-1)}\|^2)\\
+2 \Delta t \left( \tau <B(u^{(k+1)}-u^{(k)}),u^{(k+1)}-u^{(k)}>\\ 
+\Frac{1}{2}(<Au^{(k+1)},u^{(k+1)}>-<Au^{(k)},u^{(k)}>-<A(u^{(k+1)}-u^{(k)}),u^{(k+1)}-u^{(k)}>)
\right)=0
\end{array}
\right.
$$
Now, we let $E^{k+1}=\Frac{1}{2}<Au^{(k+1)},u^{(k+1)}>+
\Frac{1}{2}\|u^{(k+1)}-u^{(k)}\|^2$ and we obtain,
$$
\begin{array}{l}
2\|u^{(k+1)}-u^{(k)}\|^2+\Frac{1}{2}\|u^{(k+1)}-2u^{(k)}+u^{(k-1)}\|^2
+2\Delta t(E^{k+1}-E^{k})\\
+2\Delta t (\tau <B(u^{(k+1)}-u^{(k)}),u^{(k+1)}-u^{(k)}>-\Frac{1}{2}<A(u^{(k+1)}-u^{(k)}),u^{(k+1)}-u^{(k)}>)
\end{array}
$$
The stability is obtained when $E^{k+1}<E^k$, hence the conditions.
\end{proof}
\subsection{A ADI-RSS Scheme}

Consider the linear differential system
$$
\Frac{dU}{dt}+AU=0
$$
with $A=A_1+A_2$. Let $B_1$ and $B_2$ be preconditioners of $A_1$ and $A_2$ respectively and $\tau_1, \ \tau_2$ two positive real numbers. All the matrices are supposed to be symmetric definite positive.
We introduce the ADI-RSS schemes

\begin{eqnarray}
\Frac{u^{(k+1/2)}-u^{(k)}}{\Delta t} +\tau_1 B_1 (u^{(k+1/2)}-u^{(k)}) = -A_1 u^{(k)},\\
\Frac{u^{(k+1)}-u^{(k+1/2)}}{\Delta t} +\tau_2 B_2 (u^{(k+1)}-u^{(k+1/2)}) = -A_2 u^{(k+1/2)},
\label{RSS_ADI1}
\end{eqnarray}
and the Strang's Splitting
\begin{eqnarray}
\Frac{u^{(k+1/3)}-u^{(k)}}{\Delta t/2} +\tau_1 B_1 (u^{(k+1/3)}-u^{(k)}) = -A_1 u^{(k)},\\
\Frac{u^{(k+2/3)}-u^{(k+1/3)}}{\Delta t} +\tau_2 B_2 (u^{(k+2/3)}-u^{(k+1/3)}) = -A_2 u^{(k+1/3)},\\
\Frac{u^{(k+1)}-u^{(k+2/3)}}{\Delta t/2} +\tau_1 B_1 (u^{(k+1)}-u^{(k+2/3)}) = -A_1 u^{(k+2/3)},
\label{RSS_ADI2}
\end{eqnarray}
Of course these approach can be applied in more general situations, eg
considering $A=\displaystyle{\sum_{i=1}^m A_i}$ and $B =\displaystyle{\sum_{i=1}^m B_i}$ and the splitting
\begin{eqnarray}
\Frac{u^{(k+i/m)}-u^{(k+(i-1)/m)}}{\Delta t} +\tau_i B_i(u^{(k+i/m)}-u^{(k+(i-1)/m)}) = -A_i u^{(k+(i-1)/m)},
\label{RSS_ADIm}
\end{eqnarray}
We recall that


As a direct consequence of proposition \ref{RSS_Stab_lin}, we can prove the following result
\begin{proposition}
Under hypothesis (\ref{Hyp_H}), we have the followig stability conditions:
\begin{itemize}
\item If $\tau_i\ge \Frac{\beta_i}{2}$,$i=1,2$ the scheme (\ref{RSS_ADI1})is unconditionally stable (i.e. stable $\forall \ \Delta t >0$)
\item If $\tau_i < \Frac{\beta_i}{2}$, $i=1,2$, then the scheme is stable for
$
0<\Delta t < Min(\Frac{2}{\left(1-\Frac{2\tau_1}{\beta_1}\right)\rho(A_1)},\Frac{2}{\left(1-\Frac{2\tau_2}{\beta_2}\right)\rho(A_2)}).
$
\end{itemize} 
\label{RSS_Stab_ADI}
\end{proposition}
\begin{proof}
It suffices to apply proposition \ref{RSS_Stab_lin} to each system.
\end{proof}
\subsection{Numerical results}
USE THE MATLAB CODES IN THE DIRECTORY ADI\underline{\ }MATLAB.
\subsubsection{2D Heat Equation}
see program {\tt chaleur\underline{\ }2D\underline{\ }splitting.m}\\

\begin{figure}[!h]
\begin{center}
\includegraphics[width=7.5cm, height=7cm]{Heat_2D_strang_splitting1.png}
\includegraphics[width=7.5cm, height=7cm]{Heat_2D_strang_splitting2.png}\\
\caption{Solution of  the heat equation with $\Delta t = 0.01$, (left) and $\Delta t = 0.001$ (right) $n=31$, $\tau=1$}
\label{Strang_splitting_Heat_2D}
\end{center}
\end{figure}
The error is clearly in $\Delta t$.
\subsubsection{3D Heat Equation}
\begin{figure}[!h]
\begin{center}
\includegraphics[width=7.5cm, height=7cm]{Heat_3D_strang_splitting1.png}
\includegraphics[width=7.5cm, height=7cm]{Heat_3D_strang_splitting2.png}\\
\caption{Solution of  the 3D heat equation with $\Delta t = 0.01$, (left) and $\Delta t = 0.001$ (right) $n=31$, $\tau=1$}
\label{Strang_splitting_Heat_2D}
\end{center}
\end{figure}
The error is clearly in $\Delta t$.
%%%%%%%%%%%%%%%%%%%%%%%%%%%%%%%%%%%%%%%%%%%%%%%%%
%
%ALLEN-CAHN
%
%%%%%%%%%%%%%%%%%%%%%%%%%%%%%%%%%%%%%%%%%%%%%%%%%
\clearpage
\section{Allen-Cahn's equation}
A first inconditionnally stable scheme is (\cite{Elliott,ElliottStuart})
\begin{eqnarray}
 \Frac{u^{(k+1)}-u^{(k)}}{\Delta t} +Au^{(k+1)}+\Frac{1}{\epsilon^2}DF(u^{(k)},u^{(k+1)})=0,
 \label{ISAC1}
 \end{eqnarray}
where
$$
DF(u,v)=\left\{
\begin{array}{ll}
\Frac{F(u)-F(v)}{u-v} & \mbox{ if } u\neq v ,\\
f(u) & \mbox{ if } u= v.\\
\end{array}
\right.
$$ 
In \cite{BrachetChehabJSC} it was introduced the RSS-scheme
\begin{eqnarray}\label{RSSNLG}
\Frac{u^{(k+1)}-u^{(k)}}{\Delta t} +\tau B(u^{(k+1)}-u^{(k)})+DF(u^{(k+1)},u^{(k)})=
-Au^{(k)}
\end{eqnarray}
which enjoys of the following stability condition, see \cite{BrachetChehabJSC} for the proof.
\begin{proposition}
Under hypothsesis ${\cal H}$
\begin{itemize}
\item if $\tau \ge \Frac{\beta}{2}$, the RSS scheme is unconditionally stable,
\item if $\tau < \Frac{\beta}{2}$, the RSS scheme is stable under condition
$$
0<\Delta t < \Frac{\beta}{\rho(A)(\frac{\beta}{2}-\tau)}.
$$
\end{itemize}
\end{proposition}
%%%%%%%%%%%%%%%%%%%%%%%%%%%%%%%%%%%%%%%%%%%%%%%%%
%
%CAHN-HILLIARD
%
%%%%%%%%%%%%%%%%%%%%%%%%%%%%%%%%%%%%%%%%%%%%%%%%%
\clearpage
\section{Cahn-Hilliard's equation}
\subsection{The models}
\subsubsection{Cahn Hilliard and Patterns}
The CH equation describes the process of phase separation, by which the two components of a binary fluid spontaneously separate and form domains pure in each component. It writes as
\begin{eqnarray}
\Frac{\partial u}{\partial t} -\Delta( -\Delta u +\Frac{1}{\epsilon^2}f(u))=0,\\
\Frac{\partial u}{\partial n}=0,\\
\Frac{\partial }{\partial n}\left(\Delta u-\Frac{1}{\epsilon^2}f(u)\right)=0,\\
u(0,x)=u_0(x)
\end{eqnarray}
We have the properties
\begin{itemize}
\item Conservation of the mass: ${\bar u}=\displaytstyle{\int_{\Omega}u(x,t)dx}=\displaytstyle{\int_{\Omega}u_0(x)dx}$
\item Decay of the energy in time
$$
\Frac{\partial E(u)}{\partial t}=
-\displaystyle{\int_{\Omega}|\nabla( -\Delta u +\Frac{1}{\epsilon^2}f(u))|^2dx}\le 0
$$
\end{itemize}
A nice way to study and to simulate CH is to decouple the equation as follows:
\begin{eqnarray}
\Frac{\partial u}{\partial t} -\Delta\mu=0,\\
\mu= -\Delta u +\Frac{1}{\epsilon^2}f(u),\\
\Frac{\partial u}{\partial n}=0,\\
\Frac{\partial \mu}{\partial n}=0,\\
u(0,x)=u_0(x)
\end{eqnarray}
\subsubsection{The inpainting problem}
Cahn hilliard equations allow here to in paint a tagged picture.
Let $g$ be the original image and $D\subset \Omega$  the region of $\Omega$ in which the image is deterred. The idea is to add a penalty term that forces the image to remain unchanged in $\Omega\setminus D$ and to reconnect the fields of $g$ inside $D$. Let $\lambda >>1$
\begin{eqnarray}
\Frac{\partial u}{\partial t} -\Delta( -\epsilon \Delta u +\Frac{1}{\epsilon}f(u))&+\lambda\chi_{\Omega\setminus D}(x)(u-g)=0,\\
\underbrace{\mbox{Cahn-Hilliard equation} }&\underbrace{\mbox{Fidelity term} }\\
\Frac{\partial u}{\partial n}=0&\Frac{\partial }{\partial n}\left(\Delta u-\Frac{1}{\epsilon^2}f(u)\right)=0,\\
u(0,x)=u_0(x)&
\end{eqnarray}
Here $\chi_{\Omega\setminus D}(x)=\left\{\begin{array}{ll}1 &\mbox{ if } x\in \Omega\setminus D,\\ 0 &\mbox{ else}\end{array}$
\begin{itemize}
\item The presence of the penalization  term $\lambda\chi_{\Omega\setminus D}(x)(u-g)$ forces the solution to be close to $g$ in $\Omega\setminus D$ when $\lambda>>1$
\item The Cahn-Hilliard flow has as effect to connect the fields inside $D$
\item here $\epsilon$ will play the role of the "contrast". A post-processing is possible using a thresholding procedure.
\end{itemize}
\subsection{The RSS-Scheme}
The semi-implicit scheme
\begin{eqnarray}\label{CHC1}
\Frac{u^{(k+1)}-u^{(k)}}{\Delta t} +A\mu^{(k+1)}=0,\\
\label{CHC2}
\mu^{(k+1)}=\epsilon Au^{(k+1)}+\Frac{1}{\epsilon}f(u^{(k)}),
\end{eqnarray}
suffers from a hard time step restriction, its energy stability is guaranteed for
$$
0<\Delta<\epsilon^2
$$
see \cite{JShenACCH}
We derive the RSS-Scheme from the backward Euler's (\ref{CHC1})-(\ref{CHC2})  by
replacing $Az^{(k+1)}$ by $\tau B(z^{(k+1)}-z^{(k)})+Az^{(k)}$ for $z=u$ or $z=\mu$. We obtain
\begin{eqnarray}
\label{RSS_CH1}
\Frac{u^{(k+1)}-u^{(k)}}{\Delta t}+\tau B(\mu^{(k+1)}-\mu^{(k)}) +A\mu^{(k)}=0,\\
\label{RSS_CH2}
\mu^{(k+1)}=\epsilon \tau B (u^{(k+1)}-u^{(k)})+\epsilon Au^{(k)}+\Frac{1}{\epsilon} f(u^{(k)}).
\end{eqnarray}
We remark that this scheme preserves the steady state. We now address a stability analysis.
We first consider the linear case $(f \equiv 0)$.
\begin{theorem}
Assume that $f \equiv 0$.
If $\tau > \beta $, then the scheme (\ref{RSS_CH1})-(\ref{RSS_CH2}) 
is unconditionally stable.
\end{theorem}
\begin{proof}
We take the scalar product of (\ref{RSS_CH1}) with $u^{(k+1)}-u^{(k)}$ and of
(\ref{RSS_CH2}) with $\mu^{(k+1)}$. After the use of the parallelogram identity and usual simplifications, we obtain, on the one hand
$$
\begin{array}{l}
<u^{(k+1)}-u^{(k)},\mu^{(k+1)}>+\Frac{\Delta t \tau}{2}\left(<B\mu^{(k+1)},\mu^{(k+1)}>
-<B\mu^{(k)},\mu^{(k)}>\\
<B(\mu^{(k+1)}-\mu^{(k)}),\mu^{(k+1)}-\mu^{(k)}>\right)\\
+\Frac{\Delta t}{2}\left(<A\mu^{(k+1)},\mu^{(k+1)}>
-<A\mu^{(k)},\mu^{(k)}>\\
-<A(\mu^{(k+1)}-\mu^{(k)}),\mu^{(k+1)}-\mu^{(k)}>\right)=0,\\
\end{array}
$$
and on the other hand
$$
\begin{array}{ll}
<u^{(k+1)}-u^{(k)},\mu^{(k+1)}>&=\tau <B(u^{(k+1)}-u^{(k)}),u^{(k+1)}-u^{(k)}>\\
&+\Frac{1}{2}\left(<Au^{(k+1)},u^{(k+1)}>
-<Au^{(k)},u^{(k)}>-<A(u^{(k+1)}-u^{(k)}),u^{(k+1)}-u^{(k)}>\right)
\end{array}
$$
Taking the difference of the last two identities, we obtain 
$$
\begin{array}{l}
\{ \tau<B(u^{(k+1)}-u^{(k)}),u^{(k+1)}-u^{(k)}>-\Frac{1}{2}<B(u^{(k+1)}-u^{(k)}),u^{(k+1)}-u^{(k)}>\}\\
+\Delta t \{\Frac{ \tau}{2}<B(\mu^{(k+1)}-\mu^{(k)}),\mu^{(k+1)}-\mu^{(k)}>-\Frac{1}{2}<B(\mu^{(k+1)}-\mu^{(k)}),\mu^{(k+1)}-\mu^{(k)}>\}\\
+R^{k+1}-R^{k}=0,
\end{array}
$$
where
$$
R^{k+1}=\Frac{1}{2} <Au^{(k+1)},u^{(k+1)}>+\Delta t  <B\mu^{(k+1)},\mu^{(k+1)}>
+\Frac{\Delta t}{2}<A\mu^{(k+1)},\mu^{(k+1)}>.
$$
The scheme is then stable if $R^{k+1}<R^k$.
Hence the stability conditions.
\end{proof}
%%%%%%%%%%%%%%%%%%%%%%%%%%%%%%%%%%%%%%%%%%%%%%%%%
%
%NUMERICAL RESULTS
%
%%%%%%%%%%%%%%%%%%%%%%%%%%%%%%%%%%%%%%%%%%%%%%%%%
\section{Numerical Results}
\subsection{Implementation}
%
%Describe here the use of the cos-fft
%
The applications we are interested with are Allen-Cahn and Cahn-Hilliard equations to which homogeneous Neumann boundary conditions are associated.
We proceed as in \cite{Brachet ChehabJSC} and we first discretize in space the equation with high order finite difference compact schemes; the matrix $A$ corresponds then to the laplacien with Homogenous Neumann BC (HNBC).  Matrix $B$ is the (sparse) second order laplacian matrix with HNBC.
For a fast solution of linear systems in the RSS, we will use the cosine-fft to solve the Neumann problems with matrix $Id +\tau \Delta t B$.
{\tt test\underline{\ }Neumann\underline{\ }2D.m} is a (non RSS) solver that uses cos-fft for 2D neumann problem on the square
\subsection{Allen-Cahn equation}
{\tt Allen\underline{\ }Cahn\underline{\ }fft.m} runs (a non RSS) Allen-Cahn with semi-implicit scheme and cos-fft, see directory AC\underline{\ }CH: this seems correct\\
Also{\tt Allen\underline{\ }Cahn\underline{\ }fft\underline{\ }3D.m} runs ( a non RSS) Allen-Cahn with semi-implicit scheme and cos-fft, see directory AC\underline{\ }CH: To check

\subsection{Cahn-Hilliard equation}
USE THE (NON RSS BUT STABILIZED AS IN BERTOZZI PAPER) CODEs IN THE DIRECTORY
CH\underline{ \ }INPAINTING
%%%%%%%%%%%%%%%%%%%%%%%%%%%%%%%%%%%%%%%%%%%%%%%%%
%
%BIBLIOGRAPHY
%
%%%%%%%%%%%%%%%%%%%%%%%%%%%%%%%%%%%%%%%%%%%%%%%%%
\begin{thebibliography}{99}
\bibitem{BrachetChehabJSC} Matthieu Brachet, Jean-Paul Chehab,
Stabilized Times Schemes for High Accurate Finite
Differences Solutions of Nonlinear Parabolic Equations, J Sci Comput (2016),
DOI 10.1007/s10915--016--0223--8
\bibitem{Elliott} C.M. Elliott, The Chan-Hilliard Model for the Kinetics of Phase Separation, 
{\it in} Mathematical Models for Phase Change Problems, International Series od Numerical Mathematics, Vol. 88, (1989) Birkh\"auser.
\bibitem{ElliottStuart} C.M. Elliott and A. Stuart The global dynamics of discrete semilinear parabolic equations. SIAM J. Numer. Anal. 30 (1993) 1622--1663.
\bibitem{JShenACCH} J. Shen, X. Yang, Numerical Approximations of Allen-Cahn and Cahn-Hilliard Equations. DCDS, Series A, (28), (2010), pp 1669--1691.
\end{thebibliography}
\end{document}