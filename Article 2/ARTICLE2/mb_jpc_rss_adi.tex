 \documentclass[11pt]{article}
%\documentstyle[11pt]{article}
\usepackage{layout}
\usepackage{amssymb}
\usepackage{epsfig}
\usepackage{graphicx}
\usepackage{color}
%\usepackage{showkeys}
\usepackage{latexsym}

\setlength{\textwidth}{16.0cm} 
\setlength{\textheight}{21.0cm} 
\setlength{\evensidemargin}{ 0.5cm}
\setlength{\oddsidemargin} { 0.5cm} 
\setlength{\topmargin}{-0.5cm} \setlength{\baselineskip}  { 0.7cm}


\usepackage{amssymb}
\usepackage{algorithmicx}
\usepackage[ruled]{algorithm}
\usepackage{algpseudocode}
\usepackage{algpascal}
\usepackage{algc}

\alglanguage{pseudocode}
%%%%%%%%%%%%%%%%%%%%%%%%%%%%%%%%%%%%%%%%%%%%%%%%%%%%%%%%%%%%%%%%%%%%%%%%%%%%%%%%
%%%%%%%%%%%%%%%%%%%%

%\usepackage[counterclockwise]{rotating}
\usepackage{collectbox}

\makeatletter
\newcommand{\sqbox}{%
    \collectbox{%
        \@tempdima=\dimexpr\width-\totalheight\relax
        \ifdim\@tempdima<\z@
            \fbox{\hbox{\hspace{-.5\@tempdima}\BOXCONTENT\hspace{-.5\@tempdima}}}%
        \else
            \ht\collectedbox=\dimexpr\ht\collectedbox+.5\@tempdima\relax
            \dp\collectedbox=\dimexpr\dp\collectedbox+.5\@tempdima\relax
            \fbox{\BOXCONTENT}%
        \fi
    }%
}
\makeatother



\newenvironment{proof}[1][Proof]{\textbf{#1.} }{\ \rule{0.5em}{0.5em}}
\newcommand{\R}{I\!\!R}
\def\C{{\rm I\!C}}
\def\N{{\rm I\!N}}
\newtheorem{theorem}{Theorem}[section]
\newtheorem{acknowledgement}[theorem]{Acknowledgement}
%\newtheorem{algorithm}[theorem]{Algorithm}
\newtheorem{axiom}[theorem]{Axiom}
\newtheorem{case}[theorem]{Case}
\newtheorem{claim}[theorem]{Claim}
\newtheorem{conclusion}[theorem]{Conclusion}
\newtheorem{condition}[theorem]{Condition}
\newtheorem{conjecture}[theorem]{Conjecture}
\newtheorem{corollary}[theorem]{Corollary}
\newtheorem{criterion}[theorem]{Criterion}
\newtheorem{definition}[theorem]{Definition}
\newtheorem{example}[theorem]{Example}
\newtheorem{exercise}[theorem]{Exercise}
\newtheorem{lemma}[theorem]{Lemma}
\newtheorem{notation}[theorem]{Notation}
\newtheorem{problem}[theorem]{Problem}
\newtheorem{proposition}[theorem]{Proposition}
\newtheorem{remark}[theorem]{Remark}
\newtheorem{solution}[theorem]{Solution}
\newtheorem{summary}[theorem]{Summary}
\newcommand{\Frac}[2] {\frac{\textstyle #1} {\textstyle #2}}
\newcommand{\taumin}{\tau_{min}}
\newcommand{\taumax}{\tau_{max}}
\newcommand{\sigmin}{\sigma_{min}}
\newcommand{\sigmax}{\sigma_{max}}
 \newcommand{\halmos}{\hfill $\;\;\;\Box$\\}

\begin{document}

\title{Stabilized Splitting Schemes for Allen Cahn and Cahn Hillard equations}
\author{M. Brachet\thanks{Institut Elie Cartan de Lorraine, Universit\'e de Lorraine, Site de Metz, B\^at. A Ile du Saucy, F-57045 Metz Cedex 1, {\tt matthieu.brachet@math.univ-metz.fr}} \and J.-P. Chehab\thanks{
Laboratoire Amienois de Math\'ematiques Fondamentales et Appliqu\'ees (LAMFA), {\small UMR} 7352,
 Universit\'e de Picardie Jules Verne, 33 rue Saint Leu, 80039 Amiens France
 , ({\tt
 Jean-Paul.Chehab@u-picardie.fr}).} }


%\date{}


\maketitle

%
%



\begin{abstract}
\end{abstract}
\section{Introduction}
\section{The RSS-ADI schemes}
Let $A$ and $B$ be two $n\times n$ symmetric positive definite matrices. We assume that there exist two strictly positive constant $\alpha$ and $beta$ such that
\begin{eqnarray}
\label{Hyp_H}
\alpha <Bu,u> \le <A u, u> \le \beta <u,u>, \ \forall u \in \R^n
\end{eqnarray}
Consider the linear differential system
$$
\Frac{dU}{dt}+AU=0
$$
with $A=A_1+A_2$. Let $B_1$ and $B_2$ be preconditioners of $A_1$ and $A_2$ respectively and $\tau_1, \ \tau_2$ two positive real numbers. All the matrices are supposed to be symmetric definite positive.
We introduce the ADI-RSS schemes

\begin{eqnarray}
\Frac{u^{(k+1/2)}-u^{(k)}}{\Delta t} +\tau_1 B_1 (u^{(k+1/2)}-u^{(k)}) = -A_1 u^{(k)},\\
\Frac{u^{(k+1)}-u^{(k+1/2)}}{\Delta t} +\tau_2 B_2 (u^{(k+1)}-u^{(k+1/2)}) = -A_2 u^{(k+1/2)},
\label{RSS_ADI1}
\end{eqnarray}
and the Strang's Splitting
\begin{eqnarray}
\Frac{u^{(k+1/3)}-u^{(k)}}{\Delta t/2} +\tau_1 B_1 (u^{(k+1/3)}-u^{(k)}) = -A_1 u^{(k)},\\
\Frac{u^{(k+2/3)}-u^{(k+1/3)}}{\Delta t} +\tau_2 B_2 (u^{(k+2/3)}-u^{(k+1/3)}) = -A_2 u^{(k+1/3)},\\
\Frac{u^{(k+1)}-u^{(k+2/3)}}{\Delta t/2} +\tau_1 B_1 (u^{(k+1)}-u^{(k+2/3)}) = -A_1 u^{(k+2/3)},
\label{RSS_ADI2}
\end{eqnarray}
Of course these approach can be applied in more general situations, eg
considering $A=\displaystyle{\sum_{i=1}^m A_i}$ and $B =\displaystyle{\sum_{i=1}^m B_i}$ and the splitting
\begin{eqnarray}
\Frac{u^{(k+i/m)}-u^{(k+(i-1)/m)}}{\Delta t} +\tau_i B_i(u^{(k+i/m)}-u^{(k+(i-1)/m)}) = -A_i u^{(k+(i-1)/m)},
\label{RSS_ADIm}
\end{eqnarray}
We recall that
\begin{proposition}
Under hypothesis (\ref{Hyp_H}, we have the following stability conditions:
\begin{itemize}
\item If $\tau\ge \Frac{\beta}{2}$, the schemes (\ref{RSS_ADI1}) and (\ref{RSS_ADI2}) are unconditionally stable (i.e. stable $\forall \ \Delta t >0$)
\item If $\tau < \Frac{\beta}{2}$, then the scheme is stable for
$
0<\Delta t < \Frac{2}{\left(1-\Frac{2\tau}{\beta}\right)\rho(A)}.
$
\end{itemize} 
\label{RSS_Stab_lin}
\end{proposition}

As a direct consequence, we can prove the following result
\begin{proposition}
Under hypothesis (\ref{Hyp_H}), we have the followig stability conditions:
\begin{itemize}
\item If $\tau_i\ge \Frac{\beta_i}{2}$,$i=1,2$ the scheme (\ref{RSS_ADI1})is unconditionally stable (i.e. stable $\forall \ \Delta t >0$)
\item If $\tau_i < \Frac{\beta_i}{2}$, $i=1,2$, then the scheme is stable for
$
0<\Delta t < Min(\Frac{2}{\left(1-\Frac{2\tau_1}{\beta_1}\right)\rho(A_1)},\Frac{2}{\left(1-\Frac{2\tau_2}{\beta_2}\right)\rho(A_2)}).
$
\end{itemize} 
\label{RSS_Stab_ADI}
\end{proposition}
\begin{proof}
It suffices to apply proposition \ref{RSS_Stab_lin} to each system.
\end{proof}
\section{Numerical results}
\subsection{2D Heat Equation}
see program {\tt chaleur\underline{\ }2D\underline{\ }splitting.m}\\

\begin{figure}[!h]
\begin{center}
\includegraphics[width=7.5cm, height=7cm]{Heat_2D_strang_splitting1.png}
\includegraphics[width=7.5cm, height=7cm]{Heat_2D_strang_splitting2.png}\\
\caption{Solution of  the heat equation with $\Delta t = 0.01$, (left) and $\Delta t = 0.001$ (right) $n=31$, $\tau=1$}
\label{Strang_splitting_Heat_2D}
\end{center}
\end{figure}
The error is clearly in $\Delta t$.
\subsection{3D Heat Equation}
\begin{figure}[!h]
\begin{center}
\includegraphics[width=7.5cm, height=7cm]{Heat_3D_strang_splitting1.png}
\includegraphics[width=7.5cm, height=7cm]{Heat_3D_strang_splitting2.png}\\
\caption{Solution of  the 3D heat equation with $\Delta t = 0.01$, (left) and $\Delta t = 0.001$ (right) $n=31$, $\tau=1$}
\label{Strang_splitting_Heat_2D}
\end{center}
\end{figure}
The error is clearly in $\Delta t$.
\section{Application to the numerical solution of Phase-Fields problems}
\subsection{Allen-Cahn's equation}
\subsection{Cahn-Hilliard's equation}
\subsubsection{The RSS-Scheme}

\begin{eqnarray*}\label{CHC}
\Frac{u^{(k+1)}-u^{(k)}}{\Delta t} +A\mu^{(k)+1}=0,\\
\mu^{(k+1)}=\epsilon Au^{(k+1)}+\Frac{1}{\epsilon}f(u^{(k)},
\end{eqnarray*}
We derive the RSS-Scheme from the backward Euler's (\ref{CHC}) by
replacing $Az^{(k+1)}$ by $\tau B(z^{(k+1)}-z^{(k)})+Az^{(k)}$ for $z=u$ or $z=\mu$. We obtain
\begin{eqnarray*}
\Frac{u^{(k+1)}-u^{(k)}}{\Delta t}+\tau B(\mu^{(k+1)}-\mu^{(k)}) +A\mu^{(k)}=0,\\
\mu^{(k+1)}=\epsilon \tau B (u^{(k+1)}-u^{(k)})+\epsilon Au^{(k)}+\Frac{1}{\epsilon} f(u^{(k)}).
\end{eqnarray*}

\end{document}